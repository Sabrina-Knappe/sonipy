\documentclass[twocolumn]{aastex61}

\shorttitle{sonipy}
\shortauthors{Patton \& Levesque}

\begin{document}

\title{Mapping the Supernova-Rich Fireworks Galaxy NGC 6946}

\author[0000-0002-7640-236X]{L.~Patton}
\email{locke.patton@cfa.harvard.edu}
\affil{Department of Astronomy, University of Washington, Seattle, WA 98195 USA}
\affil{Harvard-Smithsonian Center for Astrophysics, 60 Garden St, Cambridge, MA 02138}

\author{E. M. Levesque} % [0000-0003-2184-1581]
\email{emsque@uw.edu}
\affil{Department of Astronomy, University of Washington, Seattle, WA 98195 USA}

\begin{abstract}
The galaxy NGC 6946 is unique among supernova host galaxies. It has produced ten observed core-collapse supernovae and several other massive star transients observed within the last century, marking it as the most single most prolific supernova host galaxy. Studying supernova host environments within the rare and understudied population of resolved galaxies that have hosted multiple core-collapse SNe allows us to quantify the environmental properties of stellar populations most prone to supernova progenitor production. We present spatially-resolved metallicity, radial velocity, and H$\alpha$ maps of NGC 6946, based on observations with the Apache Point Observatory 3.5m DIS spectrograph that use fifty-five longslit positions to span the face of the galaxy. Our maps encompass \textbf{NPROFILES} metallicity and emission line profiles across the face of NGC 6946, including metallicity and $H\alpha$ flux values measurements and spectra at the host sites of four supernova host sites (SN1917A, SN1939C, SN1948B, SN1968D), as well as values at several supernova remnant locations.
\end{abstract}

\section{Introduction}


% Figure: Velocity Map
\begin{figure*}
    \centering
    \includegraphics[width=\linewidth]{images/VelMap.png}
    \caption{Caption.}
    \label{fig:VelMap}
\end{figure*}

\acknowledgments

% \clearpage

\bibliographystyle{aasjournal}
\bibliography{bib}

\end{document}




% % Dados de identificação
% \title{Relatório 01\\Partida Direta de Motor de Indução Trifásico}
% \author{Nome 1, Nome 2}
% \affil{Instituto Federal do Paraná -- Campus Campo Largo\\
% 	Curso Técnico em Automação Integrado ao Ensino Médio\\
% 	Acionamentos Elétricos\\
% 	Prof. Diego Tefili}

% \begin{document}

% \maketitle        

% % Resumo de no máximo 200 palavras
% \begin{abstract}
% Este documento orienta a descrição das atividades práticas desenvolvidas em laboratório. São usados como exemplo conceitos da Aula 01 de Acionamentos Elétricos sobre partida direta de motor de indução trifásico. Nesta atividade, um motor é acionado com conexões estrela e triângulo a vazio. As correntes nominais e de partida são medidas com amperímetro analógico e comparadas entre si. Nota-se que, mesmo sem carga, as corrente em estrela são maiores. 
% \end{abstract}

% \subsection{Conexão em Triângulo}
% A Figura~\ref{triangulo} apresenta o diagrama de força da conexão em triângulo. O diagrama de comando utilizado é o mesmo da conexão em estrela, ilustrado na Figura~\ref{comando}.

% \begin{figure}[!htb]
% 	\centering\includegraphics[width=\columnwidth]{diagrama_estrela}\\
% 	\caption{Conexão do motor trifásico em estrela. }\label{triangulo}
% \end{figure}

% \section{Resultados e Discussão}

% Nesta seção, espera-se que sejam apresentadas descrições do funcionamento do circuito, medições, comparação entre os métodos utilizados, além de justificativas para o que for observado. Para todas as medições, deve ser informado em qual circuito foram colhidas, informando unidades e os equipamentos de medida utilizados para sua aferição. 

% \section{Conclusão}

% Esta seção deve ser curta, resumindo a aprendizagem obtida na aula. Nenhuma nova informação deve ser apresentada. Devem ser feitas menções aos objetivos da aula, aos principais resultados e os conceitos apreendidos com estes. Pode ser feito na forma de itens:

% \begin{itemize}
% 	\item Partida direta de motor de indução trifásico pode ser feita em estrela ou triângulo
% 	\item As correntes nominais e de partidas são maiores em triângulo.
% 	\item Cada conexão é apropriada para um nível de tensão da rede.
% 	\item Para o motor utilizado, a conexão em estrela serve para redes de 220~V ou para auxiliar a partida; enquanto que a conexão em triângulo serve para redes de 127~V apenas.
% \end{itemize}


% %%%%%%%%%%%%%%%%%%%%%%%%%%5
% % BIBLIOGRAFIA 
% % Estilo de bibliografia ABNT. Se não tiver instalado, mude para plain ou ieeetr

% %\bibliographystyle{plain} % Inclua isso se não tiver ABNTEX instalado
% \bibliography{refs}
% \begin{thebibliography}{refs}
% \bibitem{}

% \end{thebibliography}
% \end{document}